\section{Evaluation Methodology}
\label{sec:methodology}

We evaluate MPEG-DASH performance using both standard TCP and TCP Hollywood, to understand 
how the changes to the transport protocol affect quality of experience for streaming video.
Both standard TCP and TCP Hollywood use the CUBIC congestion control algorithm; both would
be impacted by any change to the variant of TCP used.

Our evaluation setup consists of a virtual environment that uses
Mininet\footnote{\url{https://mininet.org}} to create a virtual network with a modified
Linux kernel including the TCP Hollywood patches. We used our own MPEG-DASH server and client
implementation. The TCP Hollywood API was disabled when testing with standard TCP, and a
persistent HTTP/1.1 connection was used in both cases. The virtual network used included a single
client and server connected through a switch. Path characteristics were emulated at the
server interface using 
netem\footnote{\url{https://wiki.linuxfoundation.org/networking/netem}} Token Bucket
Filter traffic control, with a fixed 5Mbps line rate, a reasonable value for our test
videos, where the highest quality is 5600kbps. The test cases are divided into two
categories: (i) a loss rate of 0.2\% (random loss) with network latencies (RTT) between 50 to
400ms; and (ii) loss rates between 0 and 1\% with network latency of 100ms. Additional testing
was carried out with varying network conditions to evaluate adaptability.

\begin{table}[!t]
    \begin{tabular}{ccccc}
        \toprule
        \thead{Index} & \thead{Encoding bit-rate\\(kbps)} & \thead{Chunk bit-rate\\(kbps)} & \thead{PSNR} & \thead{SSIM} \\
        \midrule 
            1 & 1200 & 1301 & 37.220176 & 0.955177 \\
            2 & 1800 & 1941 & 39.142917 & 0.967720 \\
            3 & 2400 & 2584 & 40.478799 & 0.974580 \\
            4 & 3000 & 3228 & 41.515695 & 0.978962 \\
            5 & 3500 & 3766 & 42.226845 & 0.981527 \\
            6 & 4000 & 4304 & 42.845958 & 0.983509 \\
            7 & 5000 & 5328 & 43.868625 & 0.986314 \\
            8 & 5600 & 6030 & 44.387282 & 0.987527 \\
        \bottomrule
    \end{tabular}
    \caption{Video representations; all representations have a resolution of
             1920x1080p. PSNR and SSIM of re-encodings calculated against original Y4M reference.}
    \label{tab:testvideos}
\end{table}

We used Big Buck Bunny\footnote{\url{https://peach.blender.org}} with 8 quality levels as a video 
test sequence. The encoding bit-rates ranged from 1200 to 5600kbps with a constant resolution of 
1920x1080p. The constant resolution eliminates the need for re-scaling in objective Quality of 
Experience (QoE)
computations, which require the videos to be of the same resolution. MPEG-TS
are used to allow enable loss recovery. The use of MPEG-TS adds about a
10\% metadata overhead to the streams in our case (the use of MPEG-TS is not
essential to our approach, and any encoding or container format that is robust to packet loss
could be used). The encoding details for the different quality levels of
the test sequence are given in Table \ref{tab:testvideos}.
As TCP Hollywood is designed for low-latency applications, we use a 16 second buffer, with
a 1 second chunk duration.

To compare the performance of MPEG-DASH over standard TCP and TCP Hollywood, we use a number of 
Quality of Service (QoS) and QoE metrics: 
\begin{description}
    \item[Start-up delay] \hfill \\
        The amount of time from the start of the test until 2 seconds of video has been downloaded and demuxed.
    \item[Stall duration] \hfill \\
        The total amount of time the video stalls due to a completely empty buffer. The
        client would wait for one additional chunk to be downloaded before resuming play-out
        when a stall occurs.
    \item[Adaptability and stability] \hfill \\
        These are measured using two metrics: (i) the average bit-rate of the downloaded
        chunks, and (ii) the percentage of chunks with a downward bit-rate switch during
        play-out.
    \item[Objective QoE] \hfill \\
        To measure the level of distortion produced due to a discarded message, we use the 
        objective QoE metrics Peak Signal to Noise Ratio (PSNR) and Structural Similarity
        (SSIM). Both metrics are full-reference. We use the original Y4M sequence as
        reference and FFmpeg\footnote{\url{https://ffmpeg.org/ffmpeg-filters.html}} for the
        computation. Since PSNR and SSIM values will be lower for lower quality chunks, the
        metrics also provide a measure of quality even in the absence of discarded messages.
\end{description}



The algorithm is as follows 

\begin{algorithm}
\caption{BOLA with TCP Hollywood}\label{euclid}
\begin{algorithmic}[1]
\Procedure{DownloadStream}{}
\State $\textit{n} \gets \textit{index of current chunk}$
\State $N \gets \textit{Total number of chunks}$
\State $B \gets \textit{Maximum Buffer Length}$
\State $b_{now} \gets \textit{Current Buffer Length}$
\State $b_{p} \gets \textit{PlaceHolder Buffer}$
\While {$\text{n} < \text{N}$}
\If {$b_{now} > B$}
\State $\Delta t \gets B - b_{now}$
\State $wait (\Delta t)$
\EndIf
\State ${b_{p} \gets b_{p} + \Delta t}$
\State ${q \gets GetBolaQualityEstimate(b_p, b_{now})}$
\While {$bytes_rx < Rx_T x size(n_q)$}
\EndWhile
\EndWhile
\EndProcedure
\end{algorithmic}
\end{algorithm}
=======
These metrics were chosen since they are widely used and have been shown to
be representative of different aspects of the user experience for MPEG-DASH
video.
