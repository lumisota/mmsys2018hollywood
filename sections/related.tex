\section{Related Work}
\label{sec:related}

HTTP adaptive streaming has become the de-facto standard for on-demand video streaming.
MPEG-DASH, an International Standard, is similar to other flavours of HAS, in that the
architecture involves a client requesting discrete chunks of media over HTTP, at a rate
determined by its adaptation algorithm. HAS protocols, by virtue of their use of chunked
encodings and HTTP over TCP, introduce significant latency, and rely heavily upon
buffering to maintain quality-of-experience. However, a previous study has shown that
latency, in a local network, can be reduced to as low as 240ms\cite{bouzakaria2014overhead}. The cost of this was
higher packaging overhead, with increases of up to 13\%.

An obvious application-layer approach to reducing latency in MPEG-DASH is to reduce the
size of chunks; however, doing this leads to a significant increase in the number of HTTP
requests sent. Wei and Swaminathan \cite{wei2014low} propose using HTTP/2's server push
mechanism to reduce the number of requests, potentially only requiring a single HTTP
request across the entire stream. Xiao et al. \cite{xiao2016dash2m} take a similar
approach, but focus on mobile devices. While HTTP/2 provides server push and
multi-streaming (and so eliminates head-of-line blocking at the application-layer), using
standard TCP means that the application will still be impacted by head-of-line blocking.
In the event of loss, data will be delayed by more than one RTT \cite{mcquistin2016tcp2};
this is problematic in low-latency applications. The use of multiple simultaneous
TCP connections provides a delivery model that is analogous to a multi-streaming protocol.
However, these connections do not share flow and congestion control state, which degrades
their performance. Further, managing these connections introduces complexity at the
application-layer.

Our approach attempts to eliminate the latency associated with head-of-line blocking by
using TCP Hollywood, whose message-oriented unordered delivery model is well suited to
low-latency applications. Other transport-layer solutions include QUIC
\cite{draft-ietf-quic-transport-latest}, a UDP-based protocol with support for stream
multiplexing. Evaluations conducted over QUIC \cite{bhat:2017:not-so-quic}, without application-layer changes,
have shown that MPEG-DASH QoE is degraded. 

Neither of these approaches -- application-layer changes (e.g., using HTTP/2) or 
novel transport-layer protocols (e.g., TCP Hollywood or QUIC) -- alone is sufficient to
improve application performance. Both are required: application-layer changes are
necessary, but these must be supported by the semantics of the underlying transport
protocol. The application-layer modifications we propose here are likely to be compatible
with other multi-streaming transport-layer protocols, including QUIC.

Complementary latency-reducing approaches include a server and network-assisted variant
of MPEG-DASH, SAND \cite{thomas2015enhancing}; this is currently in development, and will form part
of the MPEG-DASH standard. SAND enhances MPEG-DASH with asynchronous network-to-network
and network-to-client quality-related message exchange. A Software Defined Networking
(SDN) based approach has been shown to improve user QoE by providing the client with
network performance and cache content information to assist the cache and bit-rate
selection while using the SDN network to optimise the caches \cite{bhat2017network}.
Additionally, Fr\"ommgen et al. \cite{frommgen2017programming} propose a programming model
for multipath TCP scheduling; the delivery model of TCP Hollywood, combined with the type
of traffic being carried, make this an interesting basis for future work. For example,
exposing message deadlines to a multipath scheduler may increase the proportion of
messages that meet their deadline.