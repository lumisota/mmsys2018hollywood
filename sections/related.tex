\section{Related Work}
\label{sec:related}

HAS is the de-facto standard for on-demand streaming in the current Internet.
MPEG-DASH, the international standard, is similar to other flavours of HAS, all of which
use a client pull-based technique for video streaming using HTTP. MPEG-DASH is a high
latency technique relying heavily on prebuffering of video. However, a previous study on
low-latency live video streaming was able to attain a latency as low as 240ms in local
network with one frame fragments. The low-latency comes at the cost of higher packaging
overhead; up to 13\% for HD video. There are several ongoing efforts for enhancing the
performance of MPEG-DASH systems by assigning a greater role to the server. A server push
based system with HTTP/2 is proposed by authors in \cite{wei2014low}, which lowers the
latency for live DASH streams in comparison to the conventional HTTP request based
approach. The researchers found that using 1 second segments greatly improved the live
latency with their scheme, while a push based scheme prevented the request explosion
problem associated with shorter segment durations. A server and network-assisted version
of DASH, known as SAND \cite{thomas2015enhancing}, is also being developed and will be
made part of the MPEG-DASH standard. SAND will enhance MPEG-DASH with asynchronous
network-to-network and network-to-client quality related message exchange that does not
interfere with regular media streaming. A Software Defined Network (SDN) assisted approach
has been shown to improve user QoE by providing the client with network performance  and
cache content information to assist the cache and bit-rate selection while using the SDN
network to optimise the caches \cite{bhat2017network}.

Our approach attempts to solve the application-layer issues that result from the latency
that standard TCP introduces at the transport-layer. Other transport-layer solutions
include QUIC \cite{draft-ietf-quic-transport-latest}, a UDP-based protocol with support
for multi-streaming. While this eliminates the head-of-line blocking introduced by TCP,
evaluations \cite{bhat:2017:not-so-quic} have shown that MPEG-DASH QoE is degraded when
QUIC is used. This shows that application-layer changes are necessary to better support
the semantics of the underlying transport-layer protocol. The application-layer
modifications we propose are likely to be applicable to QUIC. Other application-layer
approaches include the use of HTTP/2. While this provides support for multi-streaming, when used with
standard TCP it still suffers from head-of-line blocking. The use of multiple simultaneous
TCP connections provides a delivery model that is analogous to a multi-streaming protocol.
However, these connections do not share flow and congestion control state, which degrades
their performance. Further, managing these connections introduces complexity at the
application-layer.

