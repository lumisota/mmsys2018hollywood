\section{TCP Hollywood}
\label{sec:hlywd}

TCP Hollywood \cite{mcquistin2016tcp, mcquistin2016tcp2} is an unordered and time-lined
transport protocol. While maintaining wire compatibility with standard TCP, TCP Hollywood
removes two sources of transport-layer latency: in-order delivery and reliability.
In-order delivery results in head-of-line blocking: data is not made available to the
application until all earlier data has been delivered. Providing reliability over a
best-effort network requires retransmissions, adding latency while loss is detected and
retransmissions are sent. TCP Hollywood uses message-oriented semantics, and eliminates
head-of-line blocking by delivering messages to the receiver in the order that they
arrive. TCP Hollywood also relaxes standard TCP's reliability guarantee, providing
\emph{partial} reliability instead. The TCP Hollywood API allows applications to specify a
deadline for each message, after which it will no longer be sent -- this means that if a
retransmission of a lost message is unlikely to arrive before its deadline, it will not be
sent; an unexpired message will be sent instead (i.e., the retransmitted TCP segment's
payload is different to the original transmission). Making use of this functionality
requires that the application be aware of the TCP Hollywood extensions and the content
being sent, such that it can set a deadline for each message.

The scope of this work is to evaluate the impact of HAS-over-TCP Hollywood on video
quality-of-experience, while minimising change at the application-layer (i.e., maintaining
alignment with the MPEG-DASH standard). One of the core tenets of the MPEG-DASH
architecture is that the application logic is driven by the client: it determines when,
and at which rate, chunks are requested. This means that the server can be a generic HTTP
server -- much of MPEG-DASH's popularity is owed to this approach. Given that TCP
Hollywood's deadline API requires state to be held at the server, we only consider the
benefits of its unordered delivery feature in this paper. 

The benefits of TCP Hollywood alone have been analysed
\cite{mcquistin2016tcp, mcquistin2016tcp2}: eliminating head-of-line blocking significantly
reduces the latency in lossy networks. However, the impact of these savings is muted,
unless the application is structured such that it can make use of them: data sent in each
TCP Hollywood message must be independently (of other messages) useful. Under standard
MPEG-DASH, when a frame is to be played out, but has not yet arrived, the application
will stall waiting for the missing data to arrive. When using standard TCP, this is a
sensible design choice: subsequent frames are not available to the application. However,
under TCP Hollywood, the frames that arrive while the missing frame is retransmitted
may be available to be played out: this makes skipping frames a viable design choice.
Rather than waiting for the missing frame, it is better to use error concealment
techniques to minimise the impact of loss (as far as possible) and continue play-back.
However, rate adaptation algorithms have not been designed to consider loss, and must be
reevaluated when TCP Hollywood is used.

In introducing TCP Hollywood, and therefore out-of-order delivery, at the transport-layer,
we must consider two design choices at the application-layer: the data that should be
sent in each message, such that the benefits of out-of-order delivery are maximised,
and the operation of the rate adaptation algorithm when some data may be missing. This
paper will investigate these design choices.