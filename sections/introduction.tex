\section{Introduction}

% High-level: Internet video is significant, HAS protocols are a large part of that, but
% TCP isn't optimal

Video has grown to be the dominant class of traffic on the Internet in recent
years, and it is expected to grow further. This growth has been driven by a rising number
of cord-cutters: users who have switched to consuming video solely over the Internet. HTTP
Adaptive Streaming (HAS) protocols, including proprietary protocols such as Apple's HTTP Live
Streaming (HLS) and the MPEG-DASH standard, underpin much of this traffic.  HTTP uses TCP
at the transport-layer, however, which is suboptimal for applications that wish to trade-off
reliability and order for timeliness, including Internet video applications.

% Narrow down: what's our problem with standard MPEG-DASH over HTTP+TCP?

HAS protocols operate on a pull-based technique, driven by the client application. An HTTP
server provides video in discrete chunks of equal duration. Each chunk is provided in several 
different encodings, each at a different bit-rate. The client requests each chunk in turn, 
determining the
appropriate representation to request based on its rate adaptation algorithm. The client
then plays out these chunks in order, using a buffer to attempt to reduce stalling
behaviour when a chunk doesn't finish downloading in time to play. However, the application 
is bound by the reliable delivery
semantics of TCP: the application \emph{must} wait for undelivered chunks. Stalling for
undelivered chunks affects quality-of-experience, not only with the stall itself, but with
the signals that the delay provides to the rate adaptation algorithms. Transient network
issues are amplified.

% Contributions: what are we going to do/show?

In this paper, we develop an MPEG-DASH system that uses TCP Hollywood
\cite{mcquistin2016tcp,mcquistin2016tcp2}, a variant of TCP
that provides an unordered, multi-streaming delivery model. The changes made in
TCP Hollywood are intended to reduce the impact of
losses on quality-of-experience in high-delay networks. In high-delay and high-loss networks, 
adopting it for HAS results in total stall duration that is seven times lower than that of 
standard TCP.
Further, we show small improvements in start-up delay and average media bit-rate.

% What about other methods? Multiple HTTP/1.1 connections? HTTP/2?

Similar results could be achieved by using multiple simultaneous transport-layer
connections. However, these connections maintain separate flow and congestion control
states, resulting in degraded performance. Multi-streaming in HTTP/2 allows a single
transport-layer connection to be used by multiple ordered streams. Sending each chunk
within its own stream would remove application-layer head-of-line blocking, but not the
head-of-line blocking introduced by TCP. Our approach allows for a single transport-layer
connection to be used, with the benefits that come with this, while eliminating
head-of-line blocking.

% Paper structure

The remainder of this paper is structured as follows. Section \ref{sec:hlywd} briefly
introduces TCP Hollywood, and how its delivery semantics are useful for our application.
Section \ref{sec:transport} describes the changes required to our MPEG-DASH client and
server to work across TCP Hollywood. The testbed environment is described in Section
\ref{sec:methodology}, and the evaluation results are discussed in Section \ref{sec:testing}.
Finally, Section \ref{sec:related} discusses related work, and how our contribution fits
with this, while Section \ref{sec:conclusion} concludes.
